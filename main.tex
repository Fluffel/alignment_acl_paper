% This must be in the first 5 lines to tell arXiv to use pdfLaTeX, which is strongly recommended.
\pdfoutput=1
% In particular, the hyperref package requires pdfLaTeX in order to break URLs across lines.

\documentclass[11pt]{article}

% Change "review" to "final" to generate the final (sometimes called camera-ready) version.
% Change to "preprint" to generate a non-anonymous version with page numbers.
\usepackage[review]{acl}
% \usepackage{acl}

% Standard package includes
\usepackage{times}
\usepackage{latexsym}

% For proper rendering and hyphenation of words containing Latin characters (including in bib files)
\usepackage[T1]{fontenc}
% For Vietnamese characters
% \usepackage[T5]{fontenc}
% See https://www.latex-project.org/help/documentation/encguide.pdf for other character sets

% This assumes your files are encoded as UTF8
\usepackage[utf8]{inputenc}

% This is not strictly necessary, and may be commented out,
% but it will improve the layout of the manuscript,
% and will typically save some space.
\usepackage{microtype}

% This is also not strictly necessary, and may be commented out.
% However, it will improve the aesthetics of text in
% the typewriter font.
\usepackage{inconsolata}

%Including images in your LaTeX document requires adding
%additional package(s)
\usepackage{graphicx}

% If the title and author information does not fit in the area allocated, uncomment the following
%
%\setlength\titlebox{<dim>}
%
% and set <dim> to something 5cm or larger.

\title{Syntactic Alignment in Conversations with Large Language Models: Do LLMs Adapt their Syntax Over the Long Term Similar to Humans?}
% \title{Analysing Syntactic Alignment of Large Language Models in Dialogues}

% Author information can be set in various styles:
% For several authors from the same institution:
% \author{Author 1 \and ... \and Author n \\
%         Address line \\ ... \\ Address line}
% if the names do not fit well on one line use
%         Author 1 \\ {\bf Author 2} \\ ... \\ {\bf Author n} \\
% For authors from different institutions:
% \author{Author 1 \\ Address line \\  ... \\ Address line
%         \And  ... \And
%         Author n \\ Address line \\ ... \\ Address line}
% To start a separate ``row'' of authors use \AND, as in
% \author{Author 1 \\ Address line \\  ... \\ Address line
%         \AND
%         Author 2 \\ Address line \\ ... \\ Address line \And
%         Author 3 \\ Address line \\ ... \\ Address line}
% \author{Author 1 \\
%         Address line}
\author{Florian Kandra \\
  Saarland University \\
  \texttt{florian-kandra@outlook.com} \\\And
  Vera Demberg \\
  Saarland University \\
  \texttt{vera@coli.uni-saarland.de} \\\AND
  Alexander Koller \\
  Saarland University \\
  \texttt{koller@coli.uni-saarland.de} \\}

% \author{
%  \textbf{First Author\textsuperscript{1}},
%  \textbf{Second Author\textsuperscript{1,2}},
%  \textbf{Third T. Author\textsuperscript{1}},
%  \textbf{Fourth Author\textsuperscript{1}},
% \\
%  \textbf{Fifth Author\textsuperscript{1,2}},
%  \textbf{Sixth Author\textsuperscript{1}},
%  \textbf{Seventh Author\textsuperscript{1}},
%  \textbf{Eighth Author \textsuperscript{1,2,3,4}},
% \\
%  \textbf{Ninth Author\textsuperscript{1}},
%  \textbf{Tenth Author\textsuperscript{1}},
%  \textbf{Eleventh E. Author\textsuperscript{1,2,3,4,5}},
%  \textbf{Twelfth Author\textsuperscript{1}},
% \\
%  \textbf{Thirteenth Author\textsuperscript{3}},
%  \textbf{Fourteenth F. Author\textsuperscript{2,4}},
%  \textbf{Fifteenth Author\textsuperscript{1}},
%  \textbf{Sixteenth Author\textsuperscript{1}},
% \\
%  \textbf{Seventeenth S. Author\textsuperscript{4,5}},
%  \textbf{Eighteenth Author\textsuperscript{3,4}},
%  \textbf{Nineteenth N. Author\textsuperscript{2,5}},
%  \textbf{Twentieth Author\textsuperscript{1}}
% \\
% \\
%  \textsuperscript{1}Affiliation 1,
%  \textsuperscript{2}Affiliation 2,
%  \textsuperscript{3}Affiliation 3,
%  \textsuperscript{4}Affiliation 4,
%  \textsuperscript{5}Affiliation 5
% \\
%  \small{
%    \textbf{Correspondence:} \href{mailto:email@domain}{email@domain}
%  }
% }

\begin{document}
\maketitle
\begin{abstract}
This paper explores the effects of long-term syntactic alignment in Large Language Models (LLMs). Using OpenAI's GPT-4o, artificial conversations were generated, addressing a lack in existing research of long natural conversations with LLMs.
A statistical analysis on syntactic structures present in these conversations reveals that syntactic alignment occurs in LLMs over extended periods. A second analysis further explores how the process of alignment evolves throughout a conversation, showing that LLMs progressively adjust their syntax, with the largest changes occurring early on.
% giving evidence that LLMs adapt their language more conistently with increasing contex sizes compared to humans.
The results indicate that LLMs are not only influenced by the linear order in which tokens of their inputs appear, but also that its influence becomes continuously larger with increasing context lengths.
% // This suggests an implicit adaptation to prompts that goes beyond simple instruction-following.
% LLM responses are shaped not only by the vector representations of tokens in their prompts, but also by the linear order in which those tokens appear.
\end{abstract}

\section{Introduction}

Alignment in human language and communication is a widely studied process, in which people adapt to their communication partner by coordinating their behavior and language. These adaptation processes not only appear on a visual or auditory level, such as gestures, postures or the speech rate (\citealp{Holler2011mimicry}, \citealp{shockley2009coordinative}, \citealp{jungers2009speech}), but also on more underlying levels, e.g. the semantics or syntax (\citealp{BOCK1986355}, \citealp{garrod1987semanticcoord}).
Under these latter two aspects, artificial language generation has become almost indistinguishable from human language in recent years;
Large Language Models (LLMs) are optimized to produce texts that seem as coherent as human language, yet their linguistic behavioral patterns haven't been studied much. Different from humans, LLMs work on a next-token-prediction task: They don't follow a conscious effort to convey meaning, but rather model a certain probability function to generate texts. Their concrete underlying workings are unknown, as they emerge from an optimization process on large amounts of data, making it unclear which patterns they have picked up on during that training.

Although they are never explicitly guided to exhibit behavior similar to humans, do Large Language Models nonetheless exhibit syntactic alignment in their text production, similar to us?


% Bibliography entries for the entire Anthology, followed by custom entries
%\bibliography{anthology,custom}
% Custom bibliography entries only
\bibliography{custom, works}

\appendix

\section{Example Appendix}
\label{sec:appendix}

This is an appendix.

\end{document}
